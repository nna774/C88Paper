\documentclass[twocolumn,8pt,b5paper]{extarticle}

%% \Setlength{\textwidth}{17cm}
%% \setlength{\textheight}{24cm}
%% \setlength{\leftmargin}{-0.5cm}
%% \setlength{\topmargin}{-4cm}
%% \setlength{\oddsidemargin}{0cm}
%% \setlength{\evensidemargin}{0cm}

\usepackage{xltxtra}
\setmainfont{IPAPMincho}
\setsansfont{IPAPGothic}
\setmonofont{IPAGothic}
\XeTeXlinebreaklocale "ja"

\usepackage{hyperref}
\usepackage{listings}
\usepackage{verbatim}
\usepackage{biblatex}
\addbibresource{ref}
\usepackage{graphicx}
\usepackage{myjapanese}

%\usepackage{endnotes}
%\let\footnote=\endnote
\newcommand{\vs}{\vspace{\baselineskip}}

\title{Pietのコード自動生成}
\author{NoNamea 774}

\begin{document}
\maketitle

\section{なぜこんなものを……}
ここのサークルの主がコミケに申し込むというので(初サークル参加)、
とりあえず何か出してみたいなぁ(初サークル参加)と思って「じゃあペーパー書いてみます」と言ったのが2月の中旬。
これを書き始めたのが8/12。反省している。出ない神ペーパーより出る謎ペーパー、ということで書いています。
ブログ\footnote{\url{https://nna774.net/}}のいつもの記事とどう違うんだ、って話はあるけれど、
コミケで頒布してみる って事にはそれはそれで意味があるんじゃないのかな、ということで。

\section{Pietとは}
Pietというお絵かきプログラミング言語があります。
Wikipediaによると~\cite{wppiet}、
\begin{quotation}
Piet(ピエト)は、プログラミング言語であり、難解プログラミング言語のひとつである。
Ook! などのいくつかの難解プログラミング言語を開発した David Morgan-Mar がピエト・モンドリアンの作品に影響を受けて考案した言語で、文字ではなく色を組み合わせて記述する。 サンプルソースコード[1]を一見すると、抽象画のように見える。
\end{quotation}

aaaa\footnote{もへもへ}aaamimimi
\begin{itemize}
\item \url{http://ossipedia.ipa.go.jp/ipafont/}
\end{itemize}

\section{papa}
huhuhu
pap{\tt xelatex}huhu

\begin{lstlisting}[frame=shadowbox]
xelatex foo.tex
\end{lstlisting}

\newpage

Huhuhuh、
にほんご\footnote[42]{おなかすいた}。


footnote\footnote{\url{http://poaaaaaaaaaaaaaa}}


\printbibliography 

\vs
\hrule
%\theendnotes

\null\vfill
\section*{\includegraphics[width=\baselineskip,clip]{glider.png} 奥付}
\hrule\vskip.5mm\hrule\vskip3mm
\begin{tabular}{ll}
2015/8/15 & 初版発行\\
hash: & payo@payo(の次)\\
著作・発行 & NoNameA 774 (nonamea774@nnn77) \\
メールアドレス & \href{mailto:nonamea774@gmail.com}{\nolinkurl{nonamea774@gmail.com}}\\
Web & \url{https://nna774.net/}\\
Twitter & @nonamea774\\
GPG Key & 0x0C3E3AB2\\
fingerprint & 674A 287A 21D2 2431 AD8F \\
 & D328 AEF3 C3C7 0C3E 3AB2
\end{tabular}
\vskip3mm\hrule\vskip3mm

This article is licensed unser GFDL 1.3 or any later versions. And/or CC BY-SA 4.0 International.
You can get a machine-readable Transparent copy from \url{https://github.com/nna774/C88Paper}
\end{document}
